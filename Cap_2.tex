\chapter{Trabajos Relacionados}
\section{Consideraciones Iniciales}
En este capítulo se discute... 
\section{Minería Visual de Datos}
El análisis de los datos de vastos volúmenes de datos se ha vuelto extremadamente dificil. la visualizacion de la informacion pueden ayudar en el tratamiento de esta informacion 
la tarea de encontrar infomacion que esta oculta en los datos es un proceso tedioso

las tecnicas de visualizacion de informacion son importantes cuando se habla del analisis de datos estructurados y no estructurados con variada dimensionalidad
La principal ventaja de la mineria visual de datos es que el usuario se envuelve directamente en el proceso de mineria de datos.

Cuando se hace analisi de datos primero se especifica ciertos parámetros para restringir el espacio de busqueda; de esta forma la mineria de datos trabaja de forma automatica siguiendo un algoritmo, finalmente los patrones son encontrados con un algoritmo automatico de mineria de datos para luego presentar los resultados en pantalla. Para que la mineria de datos sea efectiva   es importante incluir a los humanos en el proceso de exploración de datos y combinar la flexibilidad, creatividad y el conocimiento general de la persona con la gran capacidad de almacenamiento y procesamiento de las computadoras de hoy en dia. Dado que existe gran cantidad de patrones generados por el algoritmo de mineria de datos automática en forma textual, es casi imposible para el ser huma interpretar y evaluar el patron en detalle para extraer conocimientos interesantes y caracteristicas generales. la mineria de datos visual tiene como objetivo la integracion del humano en el proceso de mineria de dato, y la aplicacion de las capacidades perceptivas de los humanos para el analisis de grandes conjuntos disponibles. la presentacion de los datos en un formulario grafico en interactivo a menudo fomenta la formacion y la validacion de nuevas hipotesis, para una buena toma de decisiones por consiguiente una buena resolucion de los problemas. Con la visualizacion obtenemos una la formacion y vision de estas hipotesis. La verificacion de estas hipotesis puede hacerse mediante la via de visualizacion, pero lo mas conveniente es que vaya respaldado con algunas tecnicas automaticas de analisis estadistico o aprendizaje maquina, 

En resumen la mineria visual de datos se compone por un lado la mineria de datos que son la generacion o descubriento de patrones a partir de un conjunto de datos basandose en tecnicas matematicas, estadisticas o de inteligencia artificial y por otro lado las tecnicas de visualizacion que son la generacion de imageneso de representaciones graficas a partir de estos datos. (VER DOCUMENTO DE CUALIFICACION PARA REDACTAR ESTA PARTE).

Hay tres enfoques comunes para integrar lo humano en el proceso de exploración de datos:
\begin{itemize}
	\item \textbf{Visualizacion Anterior.} La visualización de los datos se realiza antes de que el algoritmo de minería de datos sea ejecutado, con la interaccion se da un control total sobre el espacio de busqueda. 
	\item \textbf{Visualizacion Posterior.} El algoritmo de mineria de datos automatica se ejecuta primero para la extraccion de patrones. Estos patrones son mostrados por la visualizacion, se puede hacer recalibraciones para posetriores explraciones, esto con el objetivo de introducir otros parametros y obtener mejores resultados.
	\item \textbf{Visualizacion fuertemente Integrado.} Un algoritmo de minería de datos automática realiza un análisis de los datos, pero no produce los resultados finales. Una técnica de visualización se utiliza para presentar los resultados  del proceso de exploración de datos. La combinación de algunos algoritmos de minería de datos automáticos y técnicas de visualización permite  realizar una retroalimentacion en el proceso de exploracion. de esta forma se permite al usuario entender y llevar mejor el proceso de exploracion. De esta forma tenemos algunas ventajas como el trabajo con datos con alto contenido de ruido; no se necesita de alto conocimiento de los algoritmos matematicos o estadisticos; permite una vision cualitativa de los datos para su posterior analisis cuantitativo.
\end{itemize}
tecnicas de visualizacion son muy utiles para dar a conocer una vision de los datos
\section{Visualización}
\section{Reducción de Dimensionalidad}
		\subsection{Proyecciones Multidimensionales}
		\subsection{Métricas de medición de calidad de una proyección}
		\subsection{Técnicas de visualización Radial}
		\subsection{Radviz}
		\subsection{Star Coordinates}
\section{Consideraciones Finales}
