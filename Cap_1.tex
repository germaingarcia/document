\chapter{Introducción}

Este es el primer capítulo de la tesis. Se inicia con el desarrollo de la introducción de la tesis. Es importante que el texto utilice la tabla de abreviaturas correctamente. En el archivo abreviaturas.tex contiene la tabla de abreviaturas. Para citar alguna de ellas debes usar los comandos $\backslash$ac\{tu-sigla-aqui\}. Si es la primera vez que utilizas la sigla ella se expandirá por completo. Por ejemplo, el comando $\backslash$ac\{CMM\} va a producir: \ac{CMM}. Si más adelante repites el mismo comando sólo aparecerá la sigla \ac{CMM}. Para explorar mucho más este comando es necesario leer su manual disponible en: $http://www.ctan.org/tex-archive/macros/latex/contrib/acronym/$


\section{Motivación y Contexto}

En esta sección se va desde aspectos generales a  aspectos específicos (como un embudo). No se olvide que es la primera parte que tiene contacto con el lector y que hará que este se interese en el tema a investigar.

El objetivo de esta sección es llevar al lector hacie el tema que se va a tratar en forma específica y dejar la puerta abierta a otras investigaciones

\section{Planteamiento del Problema}

En esta sección se realiza el planteamiento del problema que queremos resolver con la tesis. Sea muy puntual y no ocupe más de un párrafo en especificar cual es el problema que desea atacar.

\section{Objetivos}

En esta sección se colocan los objetivos generales de la tesis. Máximo dos. Si necesita ampliar estos objetivos utilice la sección de objetivos específicos.

\subsection{Objetivos Específicos}

En esta sección se coloca el los objetivos específicos de la tesis, que serán aquellos que contesten a las
interrogantes de investigación.

\section{Organización de la tesis}

En esta sección se coloca cuantos capítulos contendrá la tesis y que se tratará en cada uno de
ellos en forma resumida. Dediquele un parrafo de dos o tres lineas a explicar cada capítulo.

